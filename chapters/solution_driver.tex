\section{Musterlösung Device Driver}

Legen Sie für diese Übung einen neuen Ordner\emph{driver} unter dem \emph{workspace} an.
Unter \emph{driver} soll für die nächsten zwei Übungen jeweils ein Ordner erstellt werden:
\emph{char} und \emph{block}.

\subsection{Char Driver}

Schauen Sie sich nocheinmal das \emph{Char Driver} Beispiel in der Vorlesung an und vergleichen Sie es mit der
\emph{/dev/zero} Gerätedatei. Was muss man im Beispiel von der Vorlesung ändern, um den gleichen Output wie die
\emph{zero}-Gerätedatei zu erhalten? \\

\underline{\textcolor{red}{Anstelle von "'a\textbackslash n"', muss "'0"' ausgegeben werden}\hspace{0.47\textwidth}} \\

Überprüfen Sie Ihren Vorschlag indem Sie es implementieren. Bringen Sie im ersten Schritt den Treiber von der Vorlesung
zum Laufen. Übernehmen Sie das Makefile aus den vorherigen Übungen. Kompilieren und verwenden können Sie das Modul über die bereits
bekannten Befehle:
\begin{lstlisting}
$ make
$ sudo insmod mychar.ko
\end{lstlisting}

Sobald Sie dies geschafft haben, nehmen Sie ihre Anpassungen vor.

\subsection{Block Driver}

In dieser Übung soll ein Block driver entwickelt werden. Das Listing im Abschnitt Block-Driver in der Vorlesung ist die Ausgangslage. Bringen Sie
dieses zunächst zum Laufen.
Kopieren Sie auch hier das Makefile ins Verzeichnis und ändern Sie in der Datei den Namen. Builden Sie das Modul
mit \emph{make} und starten Sie es mit \emph{insmod}. Wie gross ist die virtuelle Disk? Überprüfen Sie das mit \emph{cfdisk}.

\begin{lstlisting}
$ make
$ sudo insmod memdrive.ko
$ sudo cfdisk /dev/memdrive
\end{lstlisting}

Vergrösseren Sie nun die virtuelle Disk auf 5 MB. Wie gross müssen Sie \emph{nsectors} wählen?
Schreiben Sie die Berechnung und das Ergebnis auf. \\


\underline{\textcolor{red}{5 MB = 5242880 Bytes => nsectors = 5242880 Bytes / 512 Bytes (Block Size) = 10240}\hspace{0.1\textwidth}} \\

Partitionieren Sie nun die Disk über \emph{cfdisk}. Es soll nur eine Partitionen erstellt und die kompletten
5 MB Speicher verwendet werden. Schreiben (\emph{write}) und Beenden Sie anschliessend (\emph{quit}). Es exsitiert nun eine
neue Datei mit dem Namen \emph{/dev/memdrive0}. Formatieren Sie die Partition mit \emph{ext2}:
\begin{lstlisting}
$ sudo mkfs.ext2 /dev/memdrive0
\end{lstlisting}

Danach können Sie die Partition mounten:
\begin{lstlisting}
$ sudo mount /dev/memdrive0 /mnt
\end{lstlisting}

Erstellen Sie nun einige Dateien und Ordner unter \emph{/mnt}. Unmounten und mounten Sie das Verzeichnis anschlissend:
\begin{lstlisting}
$ sudo umount /mnt
$ sudo mount /dev/memdrive0 /mnt
\end{lstlisting}

Was passiert mit den Dateien und Ordnern?

\underline{\textcolor{red}{Die Dateien und Ordner sind immer noch vorhanden. (Speicherfreigabe erst bei \emph{memdrive\_release()})}} \\

Machen Sie dasselbe nochmals, aber starten Sie auch den Treiber neu:
\begin{lstlisting}
$ sudo umount /mnt
$ sudo rmmod memdrive
$ sudo insmod memdrive.ko
$ sudo mount /dev/memdrive0 /mnt
\end{lstlisting}

Was passiert jetzt?

\underline{\textcolor{red}{Die Datei \emph{/dev/memdrive0} exisiert nicht mehr, weil der Speicher freigegeben wurde.}\hspace{0.2\textwidth}} \\
