\section{Fallstudie VFS}

\subsection{Eintrag im \emph{procfs}}

In folgenden Übung soll eine virtuelle Datei im procfs erstellt werden. Entwickeln Sie hierfür ein Kernel-Modul,
welches einen Eintrag unter \emph{/proc/hello} erzeugt. \\

\begin{enumerate}
   \item Übernehmen Sie im ersten Schritt das \emph{main.c} und \emph{Makefile} von der \emph{Fallstudie Grundlagen}. 
   \item Ändern Sie den Namen im Makefile zu \emph{vfs\_proc}. 
   \item Fügen Sie die Headerdatei für das \emph{procfs} hinzu:
\begin{lstlisting}
#include <linux/proc_fs.h>
\end{lstlisting}
\end{enumerate}

\subsubsection{Datei erstellen}

Hier sind die zwei für uns relevanten Funktionen des \emph{procfs}:
\begin{lstlisting}
proc_dir_entry* proc_create(
         const char *name, 
         mode_t mode, 
         struct proc_dir_entry *parent, 
         const struct file_operations *proc_fops)

void remove_proc_entry(const char *name, struct proc_dir_entry *parent);
\end{lstlisting}

Die erste Funktion fügt eine Datei hinzu und die zweite entfernen den Eintrag. \emph{name} entspricht dem Dateinamen,
\emph{mode} definiert die Zugriffsrechte, \emph{parent} zeigt auf den Eintrag an dem die Datei angehängt werden soll
und mit \emph{fops} kann man die Dateioperationen angeben. \\

Wir rufen nun die \emph{proc\_create()} während der Modulinitialisierung auf und löschen den Eintrag in der Exit-Funktion.
\begin{lstlisting}
static int __init vfs_proc_init (void)
{
        hello_file = proc_create(NAME, MODE, NULL, &hello_fops);
        if (!hello_file) {
                return -ENOMEM; // not enough memory
        }

        printk(KERN_INFO "vfs_proc: init\n");
        return 0;
}

static void __exit vfs_proc_exit (void)
{
        if (hello_file) {
                remove_proc_entry(NAME, NULL);
        }

        printk(KERN_INFO "vfs_proc: exit\n");
}

module_init(vfs_proc_init);
module_exit(vfs_proc_exit);
\end{lstlisting} 

Wenn wir \emph{NULL} bei \emph{parent} übergeben, wird die Datei direkt unter \emph{/proc} erzeugt.
\emph{NAME} und \emph{MODE} definieren wir über \emph{\#define}. Der Pointer \emph{hello\_file} ist der Dateieintrag.
\begin{lstlisting}
#define NAME    "hello"
#define MODE    0666    // read+write access to user, group, others

static struct proc_dir_entry *hello_file;
\end{lstlisting}

Als nächstes müssen wir die Dateioperationen \emph{hello\_fops} definieren. Der Linux-Kernel macht das über eine Struktur
namens \emph{file\_operations}, welche eine Reihe von \emph{Callback}-Funktionen beinhaltet. Der Struktur übergeben wir
unsere eigenen Callback-Funktionen: \emph{hello\_open()}, \emph{hello\_release()}, \emph{hello\_read()} und \emph{hello\_write()}.
Kopieren Sie diese Struktur \textbf{vor} die Init-Funtkion.

\begin{lstlisting}
// callbacks for file operations
static const struct file_operations hello_fops = {
        .owner   = THIS_MODULE,     // pointer to module
        .open    = hello_open,      // callback for open()
        .release = hello_release,   // callback for close()
        .read    = hello_read,      // callback for read()
        .write   = hello_write,     // callback for write()
};
\end{lstlisting}

Die Callback-Funktionen belassen wir zunächst sehr einfach:
\begin{lstlisting}
// gets called on file open
static int hello_open(struct inode *inode, struct file *file)
{
        printk(KERN_INFO "hello: open\n");
        return 0;
}

// gets called on file close
static int hello_release(struct inode *inode, struct file *file)
{
        printk(KERN_INFO "hello: close\n");
        return 0;
}

// gets called on file read
static ssize_t hello_read(struct file* file, char __user* buf, size_t buf_size, loff_t* offset)
{
        printk(KERN_INFO "hello: read\n");
        return 0;
}

// gets called on file write
static ssize_t hello_write(struct file* file, const char __user* buf, size_t buf_size, loff_t* offset)
{
        printk(KERN_INFO "hello: write\n");
        return buf_size;
}
\end{lstlisting}

Speichern Sie nun die \emph{main.c} Datei, kompilieren Sie das Programm über \emph{make} und laden Sie das Modul über
\emph{insmod}:
\begin{lstlisting}
$ make
make -C /lib/modules/3.2.57/build M=/home/tux/workspace/vfs/procfs modules
make[1]: Entering directory
  Building modules, stage 2.
  MODPOST 1 modules
make[1]: Leaving directory
$ sudo insmod vfs_proc.ko
\end{lstlisting}

Mit \emph{rmmod} können Sie das Modul wieder entladen.
\begin{lstlisting}
$ rmmod vfs_proc
\end{lstlisting}
\clearpage
Was passiert bei \emph{ls -l /proc/hello}? \\

\underline{\hspace{\textwidth}} \\

Was passiert bei \emph{cat /proc/hello}? \\

\underline{\hspace{\textwidth}} \\

Was passiert bei \emph{echo bye > /proc/hello}? \\

\underline{\hspace{\textwidth}} \\

Was steht anschliessend im Kernellog? \\

[00000.000000] \underline{\hspace{0.5\textwidth}} \newline
[00000.000000] \underline{\hspace{0.5\textwidth}} \newline
[00000.000000] \underline{\hspace{0.5\textwidth}} \newline
[00000.000000] \underline{\hspace{0.5\textwidth}} \newline
[00000.000000] \underline{\hspace{0.5\textwidth}} \newline
[00000.000000] \underline{\hspace{0.5\textwidth}} \newline
[00000.000000] \underline{\hspace{0.5\textwidth}} \newline

\subsubsection{Dateioperationen}

Um den Inhalt der Datei zu definieren, müssen wir die \emph{hello\_read()} Funktion erweitern.
\begin{lstlisting}
static ssize_t hello_read(struct file* file, char __user* buf, size_t buf_size, loff_t* offset)
{
        // determine length of CONTENT (if buffer size < content length, then use buf_size instead)
        size_t size = min(buf_size, strlen(CONTENT));

        // if file offset is > 0, then return with 0 bytes read
        if (*offset > 0) {
                return 0;
        }

        // write CONTENT into userspace memory
        if (copy_to_user(buf, CONTENT, size) != 0) {
                return -EFAULT;
        }

        // increment offset
        *offset += size;

        printk(KERN_INFO "hello: read\n");
        return size;
}
\end{lstlisting}

Zuerst ermitteln wir die Anzahl Bytes, die wir in den Userspace schreiben. Danach überprüfen wir, mit welchen Offsets die
Datei gelesen wird. Falls nicht der Anfang der Datei gelesen wird, melden wir dem Kernel das Dateiende (return 0). Würden
wir das nicht machen, würde der Kernel die Datei unendlich weiterlesen, weil nie ein Dateiende kommt (Probieren Sie es aus!).
Der Kern unser Funktion ist der \emph{copy\_to\_user()} Aufruf. Weil wir im Kernelspace arbeiten, können wir nicht direkt
auf den Buffer \emph{buf} zugreifen. \emph{copy\_to\_user()} kopiert die Anzahl Bytes von unserem String 
\emph{CONTENT} in den Buffer \emph{buf}. Danach inkrementieren wird die Offsetvariable und geben die Anzahl geschriebenen
Bytes zurück. \\

\emph{CONTENT} definieren wir als statischen String:
\begin{lstlisting}
#define CONTENT "hello reader!\n"
\end{lstlisting}

Kompilieren und laden Sie das Modul. Vergessen Sie nicht das alte Modul zuvor mit \emph{rmmod} wieder zu entfernen. \\

Was passiert nun bei \emph{cat /proc/hello}? \\

\underline{\hspace{\textwidth}} \\


\subsubsection{Zusatzaufgabe (Optional)}

Erweitern Sie den Code so, dass die Datei beschrieben werden kann und beim anschlissenden Lesen wird der neue
Inhalt ausgegeben. Beispiel:
\begin{lstlisting}
$ cat /proc/hello
hello! hello!
$ echo bye > /proc/hello
$ cat /proc/hello
bye
$
\end{lstlisting}

Verwenden Sie dazu analog zu \emph{copy\_to\_user()} die folgende Funktion, um den Buffer in \emph{hello\_write()} zu lesen:
\begin{lstlisting}
unsigned long copy_from_user(void *to, const void __user *from, unsigned long n)
\end{lstlisting}
Schreiben Sie den Bufferinhalt in eine globale Variable. Beispiel:
\begin{lstlisting}
static char content[200];
\end{lstlisting}
