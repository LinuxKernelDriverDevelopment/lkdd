\section{Musterlösung System Call Interface}

\subsection{System Call erstellen}

In dieser Übung wollen wir nun einen eignen System Call entwickeln. Erstellen Sie einen neuen Ordner im 
Workspace mit dem Namen \emph{syscall}. Entpacken Sie den Kernel wie bereits in der \emph{Fallstudie Bootprozess}
und legen Sie diesen unter \emph{workspace/syscall/linux-source} ab. \\

Als nächstes fügen wir unseren System Call in der System Call Table ein. Editieren Sie hierzu die Datei
\emph{arch/x86/kernel/syscall\_table\_32.S} wie im Listing \ref{syscall_table2}, indem Sie \emph{.long sys\_hello}
hinzufügen. 

\begin{lstlisting}[label=syscall_table2,caption=arch/x86/kernel/syscall\_table\_32.S]
        // ..

        .long sys_sendmmsg              /* 345 */
        .long sys_setns
        .long sys_process_vm_readv
        .long sys_process_vm_writev

        // hello world system call (nr. 349)
        .long sys_hello

        // ..
\end{lstlisting} \hfill

Jeder System Call hat eine eigene Nummer, die beim Aufruf über das \emph{\%eax} Register übertragen wird. Diese Nummer
teilen Sie dem Kernel in der nächsten Datei mit (Listing \ref{syscall_nr2}). Auch die Anzahl der System Calls \emph{NR\_syscalls}
muss erhöht werden.

\begin{lstlisting}[label=syscall_nr2,caption=arch/x86/include/asm/unistd\_32.S]
// ..
#define __NR_process_vm_readv   347
#define __NR_process_vm_writev  348
#define __NR_hello              349

#ifdef __KERNEL__

#define NR_syscalls 350

// ..
\end{lstlisting} \hfill


Der Funktionsprototyp wird unter \emph{include/linux/syscall.h} definiert.
\begin{lstlisting}[caption=include/linux/syscall.h]
// ..
asmlinkage long sys_process_vm_writev(pid_t pid,
                                      const struct iovec __user *lvec,
                                      unsigned long liovcnt,
                                      const struct iovec __user *rvec,
                                      unsigned long riovcnt,
                                      unsigned long flags);
asmlinkage long sys_hello(void);

// ..
\end{lstlisting} \hfill

Nun müssen wir eine neue Datei erstellen, um die Funktion zu implementieren. Erstellen Sie folgende Datei: \emph{kernel/hello.c}
und kopieren Sie den Inhalt von Listing \ref{sys_hello2} hinein.

\begin{lstlisting}[label=sys_hello2,caption=kernel/hello.c]
#include <linux/kernel.h>
#include <linux/syscalls.h>

asmlinkage long sys_hello(void)
{
        printk(KERN_INFO "Hello system call!\n");
        return 0;
}
\end{lstlisting}

Damit die neue Datei in den Kernel kompiliert wird, müssen wird diese im Makefile angeben (Listing \ref{sys_hello_make2}).
Dafür muss nur \emph{hello.o} am Ende der \emph{obj-y} Variable angegeben werden.

\begin{lstlisting}[label=sys_hello_make2,caption=kernel/Makefile]
#
# Makefile for the linux kernel.
#

obj-y     = sched.o fork.o exec_domain.o panic.o printk.o \
            cpu.o exit.o itimer.o time.o softirq.o resource.o \
            sysctl.o sysctl_binary.o capability.o ptrace.o timer.o user.o \
            signal.o sys.o kmod.o workqueue.o pid.o \
            rcupdate.o extable.o params.o posix-timers.o \
            kthread.o wait.o kfifo.o sys_ni.o posix-cpu-timers.o mutex.o \
            hrtimer.o rwsem.o nsproxy.o srcu.o semaphore.o \
            notifier.o ksysfs.o sched_clock.o cred.o \
            async.o range.o hello.o

# ..
\end{lstlisting}

Der System Call ist nun implementiert. Um diesen zu testen, müssen wir nachfolgenden den Kernel kompilieren,
installieren und neustarten.

\begin{lstlisting}
$ make
$ sudo make install
\end{lstlisting}

\subsection{System Call aufrufen}

Im zweiten Schritt wird der System Call durch ein Userspace Programm aufgerufen. Legen Sie einen neuen Ordner \emph{syscall\_test} unter
\emph{workspace/syscall} an. Erstellen Sie eine neue Datei \emph{main.c}. \\

Wie wir in der Vorlesung gesehen haben, wird ein System Call per Interrupt ausgelöst. Anstatt den Aufruf in Assembly zu schreiben, bietet
die \emph{libc} die Funktion \emph{syscall()} an. Kopieren Sie den Code aus Listing \ref{syscall_test2} in die Maindatei.

\begin{lstlisting}[label=syscall_test2,caption=workspace/syscall/main.c]
#include <stdio.h>
#include <sys/syscall.h>

#define SYSCALL_HELLO 349

int main(int argc, char *argv[])
{
        int ret;

        printf("testing system call...\n");
        ret = syscall(SYSCALL_HELLO);
        printf("executed with return value %i\n", ret);

        return 0;
}
\end{lstlisting} \hfill

Um das Programm zu kompilieren, legen wir noch ein Makefile an:
\begin{lstlisting}[caption=workspace/syscall/Makefile]
all:
        gcc -osyscall_test main.c

clean:
        rm -f *.o
\end{lstlisting}\hfill

Kompilieren Sie nun das Programm und führen Sie es aus.
\begin{lstlisting}
$ cd ~/workspace/syscall/syscall_test
$ make
$ ./syscall_test
\end{lstlisting} \hfill

Was ist die Ausgabe? \\

\underline{\smash{\textcolor{red}{testing system call...}}\hspace{0.54\textwidth}} \\
\underline{\smash{\textcolor{red}{executed with return value 0}}\hspace{0.45\textwidth}} \\

Geben Sie die Ausgabe von \emph{dmesg | tail} an:

[00000.000000] \underline{\hspace{0.5\textwidth}} \newline
[00000.000000] \underline{\hspace{0.5\textwidth}} \newline
[00000.000000] \underline{\hspace{0.5\textwidth}} \newline
[00000.000000] \underline{\hspace{0.5\textwidth}} \newline
[00000.000000] \underline{\hspace{0.5\textwidth}} \newline
[00000.000000] \underline{\hspace{0.5\textwidth}} \newline
[00000.000000] \underline{\smash{\textcolor{red}{Hello system call!}}\hspace{0.31\textwidth}} \newline

